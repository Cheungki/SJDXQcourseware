%
% GNU courseware, XIN YUAN, 2018
%

\section{C++编程的注意点}

\frame{
\centerline{\textbf{\Huge{C++编程的注意点}}}
}

\frame{\frametitle{内存}
大的对象不要定义成局部变量,可能会引起栈溢出。
}

\frame{\frametitle{二元操作}
\begin{itemize}
\item<1-> 分配内存和释放内存。
\item<2-> 加锁和解锁。
\item<3-> 申请系统资源和释放系统资源。
\item<4-> 增加引用计数和减少引用计数。
\end{itemize}
}

\frame{\frametitle{二元操作}
利用析构方法在离开作用域时会自动调用的特性,将二元操作封装成类。
}

\frame{\frametitle{字节序}
\begin{itemize}
\item<1-> 对于需要交换的数据,如读取和写入文件的数据,网络数据包等,
要注意数据类型的大端和小端字节序的问题。
\item<2-> 需要明了装配件自身信息的字节序,以及装配件内数据常量的字节序。
\end{itemize}
}

\frame{\frametitle{字符串编码}
需要注意装配件内字符串常量的编码格式。
}

\frame{\frametitle{类的层次结构}
\begin{itemize}
\item<1-> 先前软件框架多用单根继承体系,现代观点看并不是很好。
\item<2-> 多重继承比较复杂,不容易看清逻辑,也不适合多人开发,也不是很好。
\item<3-> 尽量少用继承,多用组合。
\end{itemize}
}

\frame{\frametitle{函数或方法}
\begin{itemize}
\item<1-> 函数或方法参数要指明是in,out还是in/out。
\item<2-> 函数或方法体的开始用断言检查参数有效性。
\item<3-> 函数或方法体的返回值要注意其生命周期。
\item<4-> 函数或方法调用前检查参数的有效性,调用后检查其返回值。
\item<5-> 尽量不设计和使用递归的函数或方法,因其可能栈溢出。
\item<6-> 函数或方法体内返回错误或抛出异常时,尽量清理已完成的工作,
尽量回复到函数或方法被调用前的状态。
\end{itemize}
}

\frame{\frametitle{异常}
\begin{itemize}
\item<1-> 对于公共装配件,对外提供的函数或方法不抛异常,使用返回值指示错误。
\item<2-> 对于同一个工程中的私有装配件,对外提供的函数或方法可以抛出异常,
因为它们使用的是相同的编译器。
\end{itemize}
}

\frame{\frametitle{同步}
对于函数或方法体内静态局部变量初始化的多线程保护,
可使用double-check的技术来避免重复初始化。
对于某些形式的全局变量也可能需要使用该技术。
}

\frame{\frametitle{绑定}
\begin{enumerate}
\item<1-> 使用指针的优点和缺点。
\item<2-> 两端使用shared\_ptr共享指针的优点和缺点。
\item<3-> 两端分别使用shared\_ptr和weak\_ptr的优点和缺点。
\item<4-> 思考多线程下也能安全调用的方法。
\end{enumerate}
}

\frame{\frametitle{指针}
\begin{enumerate}
\item<1-> 智能指针引用计数的同步(CPU指令锁)存在性能损耗。
\item<2-> 参考Rust语言的内存管理,借鉴拥有者-借用者的概念。
\item<3-> 拥有者使用类自身或者unique\_ptr对象实现,借用者可实现为一个纯指针的RefPtr<T>类。
\item<4-> 静态生命周期判断暂时无法在C++语言层面做到,靠程序员自己的判断。
\item<5-> 拥有-借用-归还可以看成是Attach/Detach方法的概念化。
\end{enumerate}
}

%end
